\section{mr::UVCache Class Reference}
\label{classmr_1_1UVCache}\index{mr::UVCache@{mr::UVCache}}
{\tt \#include $<$mr\-Cache.h$>$}

\subsection*{Public Member Functions}
\begin{CompactItemize}
\item 
{\bf UVCache} (const mi\-Uint xres, const mi\-Uint yres)
\item 
{\bf $\sim$UVCache} ()
\item 
void {\bf lock} ()
\item 
void {\bf unlock} ()
\item 
void {\bf reflection} (mi\-State $\ast$const state, const mi\-Uint num\-Samples, const mi\-Scalar cosine=0.1f)
\item 
void {\bf refraction} (const mi\-State $\ast$const state, const mi\-Uint num\-Samples, const mi\-Scalar cosine=0.1f)
\item 
{\bf color} {\bf interpolate} (mi\-State $\ast$const state)
\end{CompactItemize}


\subsection{Detailed Description}
This class is used as a cache for colors based on UVs. It can be used for 'pre-baking' stuff, similar to {\bf Triangle\-Vertex\-Cache}{\rm (p.\,\pageref{classmr_1_1TriangleVertexCache})}. This class is still experimental (ie. does not work that well). 



\subsection{Constructor \& Destructor Documentation}
\index{mr::UVCache@{mr::UVCache}!UVCache@{UVCache}}
\index{UVCache@{UVCache}!mr::UVCache@{mr::UVCache}}
\subsubsection{\setlength{\rightskip}{0pt plus 5cm}mr::UVCache::UVCache (const mi\-Uint {\em xres}, const mi\-Uint {\em yres})\hspace{0.3cm}{\tt  [inline]}}\label{classmr_1_1UVCache_a0}


\index{mr::UVCache@{mr::UVCache}!~UVCache@{$\sim$UVCache}}
\index{~UVCache@{$\sim$UVCache}!mr::UVCache@{mr::UVCache}}
\subsubsection{\setlength{\rightskip}{0pt plus 5cm}mr::UVCache::$\sim${\bf UVCache} ()\hspace{0.3cm}{\tt  [inline]}}\label{classmr_1_1UVCache_a1}




\subsection{Member Function Documentation}
\index{mr::UVCache@{mr::UVCache}!interpolate@{interpolate}}
\index{interpolate@{interpolate}!mr::UVCache@{mr::UVCache}}
\subsubsection{\setlength{\rightskip}{0pt plus 5cm}{\bf color} mr::UVCache::interpolate (mi\-State $\ast$const {\em state})\hspace{0.3cm}{\tt  [inline]}}\label{classmr_1_1UVCache_a6}


\index{mr::UVCache@{mr::UVCache}!lock@{lock}}
\index{lock@{lock}!mr::UVCache@{mr::UVCache}}
\subsubsection{\setlength{\rightskip}{0pt plus 5cm}void mr::UVCache::lock ()\hspace{0.3cm}{\tt  [inline]}}\label{classmr_1_1UVCache_a2}


\index{mr::UVCache@{mr::UVCache}!reflection@{reflection}}
\index{reflection@{reflection}!mr::UVCache@{mr::UVCache}}
\subsubsection{\setlength{\rightskip}{0pt plus 5cm}void mr::UVCache::reflection (mi\-State $\ast$const {\em state}, const mi\-Uint {\em num\-Samples}, const mi\-Scalar {\em cosine} = 0.1f)\hspace{0.3cm}{\tt  [inline]}}\label{classmr_1_1UVCache_a4}


\index{mr::UVCache@{mr::UVCache}!refraction@{refraction}}
\index{refraction@{refraction}!mr::UVCache@{mr::UVCache}}
\subsubsection{\setlength{\rightskip}{0pt plus 5cm}void mr::UVCache::refraction (const mi\-State $\ast$const {\em state}, const mi\-Uint {\em num\-Samples}, const mi\-Scalar {\em cosine} = 0.1f)\hspace{0.3cm}{\tt  [inline]}}\label{classmr_1_1UVCache_a5}


\index{mr::UVCache@{mr::UVCache}!unlock@{unlock}}
\index{unlock@{unlock}!mr::UVCache@{mr::UVCache}}
\subsubsection{\setlength{\rightskip}{0pt plus 5cm}void mr::UVCache::unlock ()\hspace{0.3cm}{\tt  [inline]}}\label{classmr_1_1UVCache_a3}




The documentation for this class was generated from the following file:\begin{CompactItemize}
\item 
{\bf mr\-Cache.h}\end{CompactItemize}
